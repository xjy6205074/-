% !TeX root = ../main.tex

\chapter{NV基本介绍}


\section{NV总览}

\subsection{NV介绍}

金刚石中的氮空位缺陷(NV)目前已经被广泛运用在量子传感,量子计量,量子信息以及相关的跨学科领域中。NV之所以能获得如此深入的研究,是因为NV系统具有一些优良的特性。一是,NV的电子态容易读取和操控:电子态可以使用光泵浦来初始化;电子态可以使用微波进行相干操作;电子态可以通过荧光强度直接读出。二是,在室温情况下,NV具有极长的相干时间($T_2$)。最后,NV色心的加工技术已经非常成熟:依赖一些成熟的半导体工艺,如化学气相沉积,粒子注入,退火,我们可以优化NV色心的浓度,并且精确控制样品的杂质水平。

\subsection{NV物理结构}

NV色心是金刚石中的一种稳定缺陷;它由一个替位氮原子和最近邻的空穴组成。考虑到金刚石晶格的对称性,金刚石中会同时存在四种不同取向的NV色心。采用一般的加工方法,样品中四种不同取向的NV数目基本相同。NV发光具有方向性:对$[100]$样品,在激发光强度相同,NV深度相同的情况下,四种取向的NV荧光强度相同;不过对于$[111]$样品,同样的情况下,$[111]$取向的NV荧光要明显强于其他三种取向的NV。

NV色心主要存在两种电荷态:中性的$NV^0$态和带负电的$NV^-$态。$NV^-$态,用于缺陷从邻近的N杂质中获得了一个额外的电子而带负电。$NV^-$态具有上述提到的优良性质,因此量子传感方面研究都主要围绕$NV^-$态进行。本文以下部分,$NV$统一指$NV^-$。

\subsection{NV电子能级}

本节简单介绍NV的电子能级及光初始化和光读取的基本原理。
NV色心的能级结构见图1.2。NV色心共包含6个电子:3个来自缺陷最近邻的3个碳原子,2个电子来自缺陷的N原子,1个由边上氮杂质转移过来的电子。NV的基态是具有$\leftidx{^3}A_2$对称性的自旋三重态,分别为$m_s=0$($|0\rangle$)和$m_s=\pm{1}$  ($|\pm{1}\rangle$),其中$|0\rangle$和$|\pm{1}\rangle$间存在零场劈裂($D_{gs}=2.87$\ GHz)。当沿着NV对称轴方向施加小磁场时,$|\pm{1}\rangle$两个态间还会出现塞曼劈裂,劈裂的大小为$\Delta=2\gamma_{e} B_{axial}$,其中$\gamma_{e}$为NV的旋磁比$\gamma_{e}=2.8$\ MHz/G。

室温下NV激发态$\leftidx{^3}E$态和基态$\leftidx{^3}A_2$态的光学跃迁拥有637nm的零声子线和高展宽的声子边带(640 nm-800 nm)。在辐射跃迁中,NV的色心的电子自旋是不变量。此外,还存在非辐射跃迁通道使得NV从激发态激昂至基态($\leftidx{^3}E \rightarrow \leftidx{^1}A_1 \rightarrow  \leftidx{^1}E \rightarrow \leftidx{^3}A_2$)。激发态的$|\pm{1}\rangle$态此通道衰减至基态的$m_s=0$,此外还有激发态的$m_s=0$态此通道衰减至基态的$|\pm{1}\rangle$。但由于前一过程的强度要明显大于后一过程,所以电子由激发态经过此通道衰减至基态,自旋部分从$|\pm{1}\rangle$态转换至$|0\rangle$态。

考虑一个循环,激光泵浦将电子由基态泵浦至激发态,然后电子再由激发态衰减至基态,便可以实现自旋从$|\pm{1}\rangle$态转换至$|0\rangle$态的转换。多次重复此循环,便可以将电子自旋从混态制备到$|0\rangle$态,这就是光初始化的基本原理。

考虑同样的循环,激发过程中,$|\pm{1}\rangle$态和$|0\rangle$态吸收的光子数相同,但衰减过程中,自旋$|\pm{1}\rangle$态更容易从非辐射通道回到基态,因此在这个过程$|\pm{1}\rangle$态释放的光子比$|0\rangle$态少。因此,NV电子处在自旋$|0\rangle$态时荧光比$|\pm{1}\rangle$态强,见图1.3。利用NV荧光于态相关这个特点,可以方便地用激光读取NV的自旋态(该过程测量算符为$\sigma_z$)。

\section{NV色心的相干操作}

本节简单介绍NV色心的相干操作技术,包括连续波和脉冲波两种情况。

\subsection{连续波操作}

连续波操作用于激光功率和微波功率都比较弱的情形。考虑NV基态两个跃迁中的一个,这里以$|0\rangle \Longleftrightarrow |1\rangle$为例。当微波的频率和两个态的共振频率不相同时(微波失谐),那么微波便不能驱动两个态之间的跃迁,微波无明显作用效果。当微波共振时,微波将电子自旋从$|0\rangle$激发到$|1\rangle$态,NV的荧光下降。因此通过扫描微波频率并检测荧光强度,可以测量NV色心的共振频率。

图1.4为NV色心在弱偏置磁场下的电子顺磁共振谱线(以下简称ESR或ODMR)。由于偏置磁场的作用,$|1\rangle$态和$|-1\rangle$态简并消除,ESR谱线通常会出现两个峰,分别对应于$|0\rangle \Longleftrightarrow |\pm{1}\rangle$跃迁。特别注意的是,在微波功率和激光功率都教弱的情况下,共振峰的谱线宽度很窄,可以看到NV色心超精细结构。产生超精细结构的原因是NV中的电子于最近邻的N核间的费米接触。对$\leftidx{14}N$样品,$I=1$,谱线每一支都劈裂为3个小峰,即总共可以看到6个共振峰,如图1.4中的情况。对$\leftidx{15}N$样品,峰劈裂为2个,总共可见4个共振峰。对一些系综样品,ODMR可以出现更多的峰,原因是系综样品同时存在4种不同轴向的NV。

\subsection{脉冲操作}

使用共振微波可以实现基态下自旋态之间的转换,这实现处在不同自旋态上的布局数周期振荡,振荡频率定义为微波场的拉比频率,它于微波场的功率大小有关,如图1.5所示。

在图1.5中,画出了Rabi振荡的脉冲形式。下面以$|0\rangle \Longleftrightarrow |1\rangle$为例,简单解释该脉冲。首先,泵浦激光将电子自旋初始化到$|0\rangle$态。接下来连续微波场作用,电子自旋布局数在$|0\rangle$态和$|1\rangle$态之间振荡。最后一束探测光读取电子自旋的经微波场作用后所处的态。

本段简单介绍Rabi振荡振幅衰减的原因。理想的二能级系统的Rabi振荡不会产生衰减。而NV的Rabi振荡衰减主要由两方面的因素,一方面是拍频:由于前面提到的NV与紧邻氮核的超精细相互作用,导致同时有2个或者3个跃迁被统一束微波激发,不同跃迁由于失谐不同,对应Rabi频率也有细微区别,即产生拍频;另一方面是NV色心的非均匀展宽,这一部分将在后续章节中详细介绍。对同样的系综,通常情况下,驱动的微波场功率越强,Rabi振荡的相干时间也越长。这是由于微波场起到了NV色心与周围环境退耦合的作用。

\section{NV色心退相干过程}

NV色心主要有3种退相干过程,下面分别简单介绍。其中$T_2^*$核$T_2$过程会在后面的章节中做进一步详细的讨论。
\subsection{$T_2^*$过程}
在许多物理过程中,如ESR,Ramsey等,系综的退相干的特征时间都是$T_2^*$。简单来说,产生特征时间为$T_2^*$的退相干过程的原因是NV色心与周围噪音环境的直流耦合。对单色心NV,$T_2^*$退相干过程主要由于NV色心的电子与样品中的$\leftidx{13}C$的磁偶极相互作用。对于自然自然丰度的样品,$\leftidx{13}C$为$1.1\%$左右,$T_2^*~1-10$ us。而对于$\leftidx{13}C$纯化样品,,$T_2^*$可达100 us。
对于系综样品,$T_2^*$主要由系统的非均匀展宽导致。

$T_2^*$的测量方法主要有两种。一种是利用Ramsey脉冲序列,如图1.6所示。首先,激光将态初始化到$|0\rangle$态;接下来一束$\frac{\pi}{2}$脉冲将自旋打到$|0\rangle$态核$|1\rangle$态的叠加态上;之后,NV只与噪音环境相互作用,自由演化时间$\tau$;第二束$\frac{\pi}{2}$脉冲将自旋重新转到$|0\rangle$和$|1\rangle$测量基矢上;最后光读取自旋状态。$T_2^*$通过拟FID谱线的衰减包络获得。

另一种方法是测量系统的脉冲ESR谱线。在比较小的激光功率和微波功率的情况下,这两个因素对谱线的增宽可以忽略,谱线的极限线宽和NV的$T_2^*$有关。实验上测量ESR谱线在的半高宽(FWHMM)$\Gamma$, $T_2^*=\frac{1}{\pi \Gamma}$。

\subsection{$T_2$过程}
退相干特征时间$T_2$由自旋回波脉冲测量。脉冲见图1.7。自旋回波脉冲(spin echo)可以看作是 Ramsey脉冲的一个扩展:激光初始自旋到$|0\rangle$态,一束$\frac{\pi}{2}$脉冲将自旋打到$|0\rangle$态核$|1\rangle$态的叠加态上,自由演化时间$\tau/2$,施加用于对焦的$\pi$脉冲,第二束$\frac{\pi}{2}$脉冲将自旋重新转到$|0\rangle$和$|1\rangle$测量基矢上,光读取自旋状态。同样通过拟合衰减包络得到$T_2$。由于中间$\pi$脉冲也成为对焦脉冲。的施加,直流噪音在前后两个自由演化时间内积累的净相位为零,即NV与环境中的直流噪音退耦合,因此该$\pi$脉冲也成为对焦脉冲。

影响$T_2$主要设计两个过程,一方面还是由于NV和周围$\leftidx{13}C$的超精细耦合作用。在所加磁场不大的情况下($B \sim 10$ $\si{Gauss}$ ),测到SPIN-ECHO谱线会有两个明显特征,一是NV相干性的恢复与消失,产生的原因是$\leftidx{13}C$在偏置磁场的作用下进动,与NV耦合的噪音场具有中心频率$w_c=\gamma_{n} B$ ($\gamma_{n}$为$\leftidx{13}C$的旋磁比),谱线恢复的时间$\tau$满足$w_c \tau=4 \pi$。二是谱线包络的衰减,产生这个衰减主要是$\leftidx{13}C$团簇间的磁耦极相互作用和$\leftidx{13}C$的自旋扩散效应。

另一个影响因素是样品的顺磁自旋杂质,如样品中氮施主贡献电子。当样品中的N浓度较高时,这个效应便十分明显。

为提升系统的$T_2$,可以使用高阶的对焦脉冲,如CPMG-N,另一方面,对对焦脉冲的相位进行调制,如XY-8脉冲,在操作错误不大的情况下,可以有效抑制脉冲错误对$T_2$的影响。

\subsection{$T_1$过程}
测量$T_1$的脉冲见图1.8。首先,激光将态初始化到$|0\rangle$态,系统自由演化,最后激光读取。$T_1$的时间要远大于$T_2$,通常在$\si{ms}$水平。
$T_1$过程主要是自旋声子耦合导致的。

