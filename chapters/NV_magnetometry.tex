% !TeX root = ../main.tex

\chapter{NV磁力计}

\section{直流(DC)磁场测量}
NV色心被广泛用于磁力计的研究中。目前采用NV磁力计探测的磁场种类可以大致化为两部分。一步外部的磁场源,例如用NV磁力计测量生物细胞的磁性,一些二维材料的磁性质,电子电路等等;另一方面金刚石内部的磁源,如金刚石内部的顺磁杂质,$\leftidx{15}N$,$\leftidx{13}C$等等。

金刚石磁力计的优化也可以大致分为几个方面:一是提供磁场探测的灵敏度,限制金刚石磁力计磁场灵敏度的主要是光散粒噪音(photo shot noise)。针对这一问题,通过采用系综NV样品,参与传感的NV越多,系统的荧光越强,探测器收集到的光子越多,可以有效系统的磁场灵敏度。二是提高的系统探测的空间分辨率,单NV磁力计在这一方面取得了很大的进展,单NV磁力计可以实现nm极的空间分辨率,也有许多研究表面单NV可以直接探测金刚石表面有机体中的单核磁信号。三是提高探测的速率,目前NV色心的磁探测已从点向面扩展,大量的研究正围绕NV磁成像技术进行。在兼顾磁场探测灵敏度,空间分辨率的情况下,提高成像速度是目前研究的一个热点。

这里再次列出NV磁力计相较其他一些磁力计所具有的优势。图2.1给出了一些主流磁力计的相关参数。金刚石磁力计兼具探测灵敏度与空间分辨率。并且不同于其他一些体系,如SQUID,金刚石磁力计可以在室温下操作。并且,由于金刚石的稳定性,金刚石磁力计也可以在极端情况下使用,如高温,高压等等。

下面具体介绍金刚石磁力计的基本原理。
\subsection{ESR法}

\subsubsection{ESR法简介}
ESR法是最简单的测量直流磁场的一个方法。在第一章中提到,当施加外磁场时,NV基态$|\pm{1}\rangle$态简并消失,两个跃迁$|0\rangle\Longleftrightarrow|\pm{1}\rangle$的跃迁频率之差正比磁场在NV对称轴方向的投影量。简单解释一下,由于金刚石$D_{gs}=2.87 \si{GHz}$远大于外加磁场的等效频移,所有外加磁场的在NV对称轴的垂直方向的投影量在$D_{gs}$的作用下,无零阶和一阶效应。

测量ESR谱的方法主要有两种,除了前面提到的连续波法,还可以使用脉冲波法。脉冲波法虽然在操作上较连续波略微复杂,但是,脉冲波法,由于将探测和传感两个过程有效分离,激光对谱线增宽的效果可以忽略不计,探测灵敏更高。
\subsubsection{ESR法灵敏度}
这里直接给出ESR法测量的灵敏度,ESR法的测量的灵敏度$\eta_{ESR}$,
\begin{equation}
    \eta_{ESR}=P_F\frac{h}{g \mu_{B}}\frac{\Delta \nu \sqrt{t_m}}{\alpha \sqrt{\beta}}
\end{equation}
式子中,$P_F$为一常量,如果共振峰是高斯线形的,$P_F=\sqrt{\frac{e}{8 ln2}}$;对洛伦兹线形,$P_F=\frac{4}{3\sqrt{3}}$。$\alpha$为共振峰的对比度,$\beta$为单次测量收集到的光子数,$t_m$为单次测量的总时间。增加微波功率和激光功率可以提高谱线的对比度$alpha$,但也同样会增加谱线的线宽。通过选择合适的激光功率,以及微波功率,即$\pi$脉冲的时长,系统可以获得的最佳灵敏度如下:
\begin{equation}
    \eta_{ESR}=P_F\frac{\hbar}{g \mu_{B}}\frac{1}{\alpha \sqrt{\beta T_2^*}}
\end{equation}
该式子中,可以看出NV磁力计的灵敏度主要取决于三个因素,NV的自旋弛豫时间$T_2^*$,NV样品的亮度,以前共振谱线的对比度。后两个因素和NV距离样品的距离,和样品表面微纳结构相关。
\subsection{Ramsey法}
\subsubsection{Ramseya法简介}
前面提到可以用脉冲ESR来替换连续波ESR,来消除激光功率增宽对探测灵敏度的影响,但是不能消除微波功率增宽的影响。不过这正是Ramsey法的优势所在,它可以同时消除激光功率增宽和微波功率增宽的影响。

Ramsey脉冲已经在第一章中进行介绍,这里直接介绍磁场探测的过程。利用Ramsey法测量磁场具体有两种方法。

第一种方法是扫描自由进动时间$\tau$。首先为了消除跃迁$|0\rangle\Longleftrightarrow|\pm{1}\rangle$的简并性,我们给样品施加一个偏置磁场,通常选择在$500 \si{Gauss}$附近,这样还可以通过光极化效应,极化NV最近邻的N核,使得跃迁的超精细劈裂消失。此时,选择其中的一支跃迁作探测,用作操作$\pi/2$脉冲的微波的频率都与该共振峰保持同步,这可以通过测量ODMR谱来实现微波的对焦。接下,让样品于需要测量的小磁场进行耦合,在自由进动阶段,样品会积累一个直流相位$\phi=2 \pi \gamma_{e} B \tau$,第二束微波脉冲将积累的相位转化布居数差,最后通过激光读出。最后拟合曲线的振荡频率,可以反推待测量场的大小。

第二种方法是固定自由进动时间$\tau$,这时$\tau$同常选择在最佳值$\tau~T_2^*/2$,当待测磁场有变化时,由于信号积分时间$\tau$保持不变,所以积累相位的变化只与待测场有关,通过荧光变化的大小便可以推知待测磁场的变化。
\subsubsection{Ramseya法灵敏度}
Ramsey法测量直流场的灵敏度$\eta_{Ramsey}$可以由下面的式子给出,这里 针对的是第二种探测方法:
\begin{equation}
    \eta{Ramsey}=\frac{\hbar}{g \mu_{B}}\frac{1}{\sqrt{\tau}}\frac{1}{\alpha \sqrt{\beta}}
\end{equation}
将最优探测时间代入,可以得到最佳的测量灵敏度为:
\begin{equation}
    \eta{Ramsey}=\frac{\hbar}{g \mu_{B}}\frac{1}{\alpha \sqrt{\beta T_2^*}}
\end{equation}
同样灵敏度受到样品的$T_2^*$限制。

\section{交流(AC)磁场测量}
\subsection{AC场测量方法简述}
交流磁场可以通过自旋回波脉冲测量。前面提到过,自旋回波可以将系统的相干时间从$T_2^*$提升到$T_2$,因此可以提升传感灵敏度。但自旋回波序列无法用于直流磁场的探测。原因是中间对焦$\pi$脉冲的施加,导致直流场整体积累的相位为零,导致自旋回波脉冲测量对直流场不敏感。自旋回波脉冲系统相干时间提升到$T_2$也是同样的原因,系统与直流噪音场退耦合。

设想一个自旋回波脉冲,其总信号积分时间为2$\tau$,那么,他对于频率为$f_{ac}=1/\tau$的交流磁场非常敏感。同样的,如果由于交流磁场的存在,系统在进动时间内积累的总相位$\phi=4\gamma_{e} B \tau$。选择合适的传感时间,类似于Ramsey法的后一种测量方式,测量得到的谱线的振荡频率于待测交流磁场的幅度有关,由此可以测量得到的交流磁场的大小。
\subsection{AC场测量灵敏度}
自旋回波法测量交流场的灵敏度$\eta_{AC}$可以由下面的式子给出,
\begin{equation}
    \eta{AC}=\frac{\pi \hbar}{2 g \mu_{B}}\frac{1}{\sqrt{\tau}}\frac{1}{\alpha \sqrt{\beta}}
\end{equation}
将最优探测时间代入,可以得到最佳的测量灵敏度为:
\begin{equation}
    \eta{Ramsey}=\frac{\pi \hbar}{2 g \mu_{B}}\frac{1}{\alpha \sqrt{\beta T_2}}
\end{equation}
灵敏度受到样品的$T_2$限制。另外通过施加高阶脉冲,CPMG-N等,可以 提升$T_2$。

\section{NV对于稳定场测量的讨论}
在前面的三个节中,讨论的都是NV磁力计对于一个完全确定的物理场的测量。即对直流磁场,他有确定大小;对交流场,信号有完全确定的频率,相位,和幅度。最后所得的灵敏度的表达式都具有以下形式:
\begin{equation}
    \eta \propto \frac{1}{\alpha \sqrt{\beta T}}    
\end{equation}
即灵敏度都取决于系统的相干时间,信号对比度,以及荧光计数。这个公式对我们优化测量过程中激光功率,微波功率,NV深度的选择等等参数都具有指导性的意义。

第二点需要强调的是上述式子定义的灵敏度都具有单位$\si{T}/\sqrt{\si{Hz}}$。灵敏度还对时间作了归一。这主要原因是限制灵敏度的photo shot noise,该噪音反比于测量获得的光子数的$1/2$次方。测量时间越长,得到的光子数越多,其灵敏度越好,但这于系统的特征无关,所以对时间取了归一。另外,也有工作对传感的区域大小也进行归一,灵敏度具有单位$T \cdot {\si{Hz}}^{-1/2} \cdot {\si{\mu m}}^{-3/2}$。

\section{磁噪音谱测量}
\subsubsection{测量方法简介}
前面讨论的都是NV磁力计对完全确定场的测量,即场的自相关函数为1。这部分讨论NV磁力计测量噪音场。为了足够的频率分辨率,对噪音谱的测量需要选择高阶的退耦脉冲。
退耦脉冲作用下,系统的相干性可以统一用下列公式来描述,$C(t)=e^{-\xi(t)}$,其中$\xi(t)$于与施加脉冲的滤波函数$F_t(w)$以及噪音场的功率谱$S(w)$有关,具体有以下形式:
\begin{equation}
    \xi (t)=\frac{1}{\pi}\int_{0}^{\infty}\dif w S(w) \frac{F_t(w)}{w^2}
\end{equation}

测量噪音场即是测量与系统所耦合的噪音场的功率谱函数$S(w)$,在理想情况下,
\begin{equation}
    \frac{F_t(w)}{w^2 t}=\delta (w-w_0)
\end{equation}
即退耦脉冲等效于一个$\delta$调制,则,
\begin{equation}
    \xi (t)=\frac{1}{\pi} t S(w_0)
\end{equation}
功率谱可以由此求出,
\begin{equation}
    S(w_0)=-\pi \ln {(C(t))}/t
\end{equation}
上述公式中出现了时间的一次方,这是针对噪音场为白噪音。对于$1/f$噪音和$1/f^2$噪音,相应的要修改为时间的平方和时间的三次方。
对于理想的CPMG脉冲,即对焦脉冲都是完美的,则滤波函数有以下的表达式,
\begin{equation}
F_N^{CPMG}(wt)=8\sin^2(\frac{wt}{2})\frac{\sin^4(\frac{wt}{4n})}{\cos^2(\frac{wt}{2n})}    
\end{equation}
滤波函数的中心频率位于$w_0=\pi n/t$。

在第三章中,会详细讨论系统滤波函数的形式,尤其是操作不完美的情况,并且获得的滤波函数表达对于第四章定义组合脉冲的评价函数十分重要。
