% !TeX root = ../main.tex

\begin{acknowledgements}

在研究学习期间,我首先非常感谢我导师孙芳稳教授,感谢他对我持续的支持,鼓励与耐心。他对NV色心以及相关交叉学科的知识与见解为我的研究提供了诸多的帮助。我非常庆幸可以选择孙芳稳教授作为我的导师。首先,孙老师非常地慷慨地允许我作为一个本科加入他的实验室进行科学研究。在我最初涉足这个领域时,一切对我来说都十分陌生,不过孙老师提供了全方位的帮助。他给我提供了丰富的参考文献,介绍从事科研的工作的思想方法,并且安排了手下得力的研究生对我进行辅导。在工作过程中,孙老师鼓励我们本科生在组会上发言,经常关关心我们的工作进展,并提供指导。在工作发表中,孙老师在文章修改上提出了宝贵的意见。除此之外,本科中有许多的奖学金答辩,孙老师都毫不保留地向我分享了做报告的经验,并且亲自帮助我修改PPT,这些帮助对我的答辩起到了很大的作用。在我的海外留学申请中,孙老师也给我分享了他的海外留学经验。

感谢对我的帮助很多的董扬学长,对我来说他是一个亦师亦友的人。一方面董扬师兄在实验上对我言传身教,我至始至终都记得他给我教导的“脑子比困难转得快”,“实验前想尽可能想透可能遇到的问题”。这些指导对我的实验起到了很大的帮助。师兄的理论水平极高,在他的指导下,我也尝试做了一些很有意义理论性的工作。其次,师兄把最新的思想带给我,这些思想包括怎么创新,也包括做人处事的,这种不仅授人与鱼,更授人以渔的做法让我感到。本科阶段,师兄也帮助我扩充世界观,最近看了师兄分享的一些剧,包括“硅谷”,“奔腾人生”,“胜利即正义”,极大地扩充了我的世界观。总结来说,师兄对我来说更是一个人生的导师。

我感谢实验室其他师兄师姐对我的帮助。李翠红师姐教我了共聚焦光路的基本知识,以及spin-core的使用方法。杜博师姐给我讲述了g2的原理以及测量方法。张少春师兄教了我磁场的调整。李登峰师兄对样品上的帮助。很感谢他们在这段时间对我关心与帮助。

除此之外,我感谢Dmitry Budker教授,他慷慨地接纳了加入他的研究小组进行暑期科研。在Budker教授的实验室中,我学习了先进的实验室自动化技术,直接参与到了高水平的实验过程中,更深刻体会了小组协同合作的精神。短短的两个月,在Budker教授和实验室师兄师姐的帮助下,我们完成了零场磁力计的相关研究工作。

最后,我感谢我本科阶段,父母对我的持续支持与鼓励;班主任老师对我学业和生活上的关心;室友班级同学对我的帮助。


\end{acknowledgements}
