% !TeX root = ../main.tex

\begin{abstract}
    目前固态电子自旋缺陷作为一个的平台,广泛运用与量子信息和量子计量学的研究。目前广泛的研究的体系有金刚石中的氮空位缺陷,硅空位缺陷,以及锗空位缺陷。其中金刚石中的氮空位缺陷(NV),由于其优良的系统特性,包括在室温下的长相干时间,量子态的光学可操作性及光学可读性,被广泛用于量子磁力计的搭建以及相关跨学科研究。基于系综NV的磁力计较基于单NV的磁力计有一些明显优势。一方面,基于系综NV的磁力计具有更高的探测灵敏度;另一方面基于系综NV的磁力计在磁场成像方面具有明显的优势。基于系综NV的磁力计,由于采用简单的平脉冲很难实现对所有参与传感自旋的的相干操作(量子传感的基本要求之一),而限制了这类磁力计的性能。而组合脉冲可以有效的解决上述问题。本文主要讨论如何设计相应的组合脉冲。
    
    
    本文先简单介绍NV的色心的自旋和光学性质,之后本文介绍基于NV色心的磁探测协议,并讨论影响系统探测灵敏度的几个主要因素;接下来本文研究NV色心在退相干脉冲作用下,自旋系统的退相干特性,尤其是退相干脉冲存在误差的情况;最后本文研究组合操作脉冲的设计方法,比较了几种评价函数的效果。
    
  \keywords{金刚石;氮空位缺陷;退相干;相干操作;容错脉冲}
\end{abstract}

\begin{enabstract}
  Solid-state electronic spin defects, such as nitrogen-vacancy (NV), silicon-vacancy and germanium-vacancy centers in diamond, are of great interest in the study of quantum metrology. Among them, Nitrogen-Vacancy defects (NV) in diamond are believed to be the most reliable one, owing to its long coherence time at room temperature, ability of optical initialization and optical read-out. NV ensembles-based magnetometer has some significant advantages over the single-NV-based magnetometer, including higher magnetic field sensitivity and higher sensing speed especially in magnetic field imaging case.  However, it is hard to realize coherent control of all the sensor spins with plain pulse, caused by the inhomogeneous broadening and control errors of sensor spins.  Recently, composite-pulses have been put forward in solving these challenges. In this thesis, we discuss the design of robust composite pulses.
  
  We first introduce, review the optical and spin propriety of NV in diamond. We next demonstrate the the protocols for magnetic sensing and discuss  several main factors affecting the sensing sensitivity. We then present the decoherence study of NV under focusing pulses, especially we discuss the case when the control pulse is not perfect. Finally, We study the design method of composite operation pulse, and compare the several brand new evaluation functions.

 
  \enkeywords{Diamond; Nitrogen-Vacancy; Coherence; Coherent Control; Fault-tolerant composite pulses}
\end{enabstract}
